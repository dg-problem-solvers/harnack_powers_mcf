\documentclass{amsart}
\usepackage[ocgcolorlinks,linktoc=all]{hyperref}
\hypersetup{citecolor=blue,linkcolor=red}
\newtheorem{theorem}{Theorem}
\newtheorem*{thmA}{Theorem}
\newtheorem*{thmB}{Theorem}
\newtheorem*{rem}{Remark}
\newtheorem*{thmmain}{Theorem}
\newtheorem{lemma}[theorem]{Lemma}
\newtheorem{proposition}[theorem]{Proposition}
\newtheorem*{propmain}{Proposition}
\newtheorem{corollary}[theorem]{Corollary}
\theoremstyle{definition}
\newtheorem{definition}[theorem]{Definition}
\newtheorem{example}[theorem]{Example}
\newtheorem{xca}[theorem]{Exercise}

\theoremstyle{remark}
\newtheorem{remark}[theorem]{Remark}

\newcommand{\abs}[1]{\lvert#1\rvert}
\numberwithin{equation}{section}

\newcommand{\blankbox}[2]{%
  \parbox{\columnwidth}{\centering
    \setlength{\fboxsep}{0pt}%
    \fbox{\raisebox{0pt}[#2]{\hspace{#1}}}%
  }%
}

\begin{document}

\title[]
 {Harnack estimates for curvature flow equations on the sphere}

\curraddr{}
\email{}
\date{\today}

\dedicatory{}
\subjclass[2010]{}
\keywords{}

\begin{abstract}
%We prove differential Harnack inequalities for weakly convex hypersurfaces on the unit sphere shrinking by positive powers of homogeneous degree one, inverse-concave functions of principal curvatures, $f^{\alpha}>0,~\alpha\geq 1$ . As an application, we prove that the only convex, ancient solutions of curvature flows with $f$ concave and inverse-concave velocities are shrinking geodesic spheres.
\end{abstract}

\maketitle
We consider the evolution of a hypersurface $M^n$ by
\[\partial_tx=-\varphi(f)\nu,~ x:M^n\times[0,T)\to \mathbb{S}^{n+1},\]
where $f$ is a positive, smooth, homogeneous degree one, inverse-concave function of principal curvatures, and $\varphi(y)=y^{\alpha}$ and $\alpha\geq1.$
\[M_t:=x(M^n,t)\]
and $\nabla$ is the Levi-Civita connection of $M_t.$
In the sequel, $\lambda =1$ is the sectional curvature of the unit sphere.
%\begin{thmmain}[Harnack estimate] Suppose $M_t$ is a weakly convex solution on the time interval $[0,T)$. for any $t>0$ we have
%\[\partial_t\varphi-b^{ij}\nabla_i\varphi\nabla_j\varphi-\lambda\varphi\varphi^{ij}g_{ij}+\frac{\alpha \varphi}{(1+\alpha)t}\geq 0.\]
%\end{thmmain}
%\begin{thmmain}[Classification of ancient solutions]
%The only ancient, weakly convex, embedded solutions of curvature flows with $f$ both concave and inverse-concave function of principal curvatures are equators and shrinking geodesic spheres.
%\end{thmmain}
Let us set $\varphi^{ij}=\varphi'f^{ij}$ and $\varphi^{ij,kl}=\varphi''f^{ij}f^{kl}+\varphi'f^{ij,kl}$ and  $\Box:=\varphi^{ij}\nabla_i\nabla_j.$
\begin{lemma}
The following evolution equations and commutator identities hold
\begin{enumerate}
  \item $\partial_tg_{ij}=-2\varphi h_{ij}$
  \item $\partial_tg^{ij}=2\varphi g^{im}g^{jn}h_{ij}=2\varphi h^{ij}$
    \item $\partial_t h_{ij}=\nabla_j\nabla_i\varphi-\varphi(h^2)_{ij}+\lambda \varphi g_{ij}$
  \item $\partial_t h_i^j=\nabla^j\nabla_i\varphi+\varphi(h^2)_i^j+\lambda \varphi\delta_i^j$
  \item \begin{align*}
\partial_t h_{ij}=&\Box h_{ij}+\varphi^{kl}h_{rk}h_l^rh_{ij}-\varphi'fh_{ri}h_j^r-\varphi h_i^kh_{kj}\\
&+\varphi^{kl,rs}\nabla_ih_{kl}\nabla_jh_{rs}\\
&+\lambda \{(\varphi+\varphi'f)g_{ij}-\varphi^{kl}g_{kl}h_{ij}\}
\end{align*}
  \item \begin{align*}
\partial_t h_i^j=&\Box h_i^j+\varphi^{kl}h_{rk}h_l^rh_i^j-\varphi'fh_{ri}h^{rj}+\varphi h_i^kh_k^j\\
&+\varphi^{kl,rs}\nabla_ih_{kl}\nabla^jh_{rs}\\
&+\lambda \{(\varphi+\varphi'f)\delta_i^j-\varphi^{kl}g_{kl}h_i^j\}
\end{align*}
  \item $\partial_t \varphi=\Box \varphi+\varphi\varphi^{ij}(h^2)_{ij}+\lambda \varphi\varphi^{ij}g_{ij}$
  \item \begin{align*}
(\partial_t\Box-\Box\partial_t)\varphi=&\varphi^{ij,kl}\nabla_i\nabla_j\varphi(\nabla_k\nabla_l\varphi+\varphi(h^2)_{kl}+\lambda \varphi g_{kl})\\
&+2\varphi\varphi^{ik}h_k^j\nabla_i\nabla_j\varphi+2\varphi^{ik}h_k^j\nabla_i\varphi\nabla_j\varphi+(1-\alpha)\varphi|\nabla \varphi|^2
\end{align*}
\item \begin{align*}
(\nabla_i\Box-\Box\nabla_i)\varphi=&\varphi^{kl,mn}\nabla_k\nabla_l\varphi\nabla_ih_{mn}\\
&+\varphi^{kl}(h_{il}h_k^m-h_{kl}h_i^m+\lambda(g_{il}g_{kp}-g_{kl}g_{ip})g^{pm})\nabla_m\varphi.
\end{align*}
\end{enumerate}
\end{lemma}
\begin{lemma}Define $\beta:=\partial_t\varphi-\lambda \varphi\varphi^{ij}g_{ij}.$ Under the flow we have
\end{lemma}
\begin{proof}
\begin{align*}
\partial_t\beta=&\partial_t(\Box \varphi+\varphi\varphi^{ij}(h^2)_{ij})\\
=&\Box\beta+\varphi^{ij,kl}\nabla_i\nabla_j\varphi(\nabla_k\nabla_l\varphi-\varphi(h^2)_{kl}+\lambda \varphi g_{kl})\\
&+2\varphi\varphi^{ik}h_k^j\nabla_i\nabla_j\varphi+2\varphi^{ij}h_j^k\nabla_i\varphi\nabla_k\varphi+(1-\alpha)\varphi|\nabla \varphi|^2\\
&+\varphi^{ij}(h^2)_{ij}\beta+\varphi\partial_t(\varphi^{ij}(h^2)_{ij})\\
&+\lambda\{\varphi^{ij}g_{ij}\beta+\varphi\Box(\varphi^{ij}g_{ij})+2\varphi^{kl}\nabla_k\varphi\nabla_l(\varphi^{ij}g_{ij})\}.
\end{align*}
\end{proof}
 Define $\gamma:=\Box \varphi+\varphi\varphi^{ij}(h^2)_{ij}-b^{ij}\nabla_i\varphi\nabla_j\varphi=\beta-b^{ij}\nabla_i\varphi\nabla_j\varphi.$ To prove the main theorem, we will apply the maximum principle to $t\gamma+\frac{\alpha\varphi}{\alpha+1}.$
\begin{lemma}Under the flow we have
\begin{align*}
\partial_t (t\gamma+\frac{\alpha\varphi}{\alpha+1})=&\Box (t\gamma+\frac{\alpha\varphi}{\alpha+1})+\left\{\frac{(\alpha+1)\gamma}{\alpha\varphi}+\varphi^{ij}(h^2)_{ij}+\lambda\varphi^{ij}g_{ij}\right\}(t\gamma+\frac{\alpha\varphi}{\alpha+1})\\
&+t\left\{\varphi^{mn,pq}+2\varphi^{mq}b^{np}-\frac{\alpha+1}{\alpha}\frac{\varphi^{mn}\varphi^{pq}}{\varphi}\right\}\eta_{mn}\eta_{pq}+t\lambda R,
\end{align*}
where $\eta_{mn}=\nabla_m\nabla_n\varphi+\varphi(h^2)_{mn}-b^{rs}\nabla_r\varphi\nabla_sh_{mn}$, and
\begin{align*}
R:=&\varphi\Box(\varphi^{ij}g_{ij})+2\varphi^{ij}\nabla_i\varphi\nabla_j(\varphi^{kl}g_{kl})+\varphi \varphi^{ij,kl}g_{kl}\nabla_i\nabla_j\varphi\\
&-\varphi^{kl}g_{kl}b^{ij}\nabla_i\varphi\nabla_j\varphi+b^{im}b^{jn}((\alpha+1)\varphi g_{mn}-\varphi^{kl}g_{kl}h_{mn})\nabla_i\varphi\nabla_j\varphi\\
&-2b^{ij}\varphi^{kl}(g_{il}g_{kp}-g_{kl}g_{ip})\nabla^p\varphi\nabla_j\varphi-2\varphi b^{ij}\nabla_i\varphi\nabla_j(\varphi^{kl}g_{kl})\\
&+\varphi^2\varphi^{ij,kl}g_{kl}(h^2)_{ij}+2\alpha\varphi^3.
\end{align*}
\end{lemma}

\bibliographystyle{amsplain}\begin{thebibliography}{10}
\end{thebibliography}
\end{document}
