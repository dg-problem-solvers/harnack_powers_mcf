\documentclass{amsart}

%\documentclass[10 pt]{amsart}






\numberwithin{equation}{section}

\usepackage{cleveref}
\crefname{lemma}{Lemma}{Lemmata}




%Symbols
\renewcommand{\~}{\tilde}
\renewcommand{\-}{\bar}
\newcommand{\bs}{\backslash}
\newcommand{\cn}{\colon}
\newcommand{\sub}{\subset}

\newcommand{\N}{\mathbb{N}}
\newcommand{\R}{\mathbb{R}}
\newcommand{\Z}{\mathbb{Z}}
\renewcommand{\S}{\mathbb{S}}
\renewcommand{\H}{\mathbb{H}}
\newcommand{\C}{\mathbb{C}}
\newcommand{\K}{\mathbb{K}}
\newcommand{\Di}{\mathbb{D}}
\newcommand{\B}{\mathbb{B}}
\newcommand{\8}{\infty}

%Greek letters
\renewcommand{\a}{\alpha}
\renewcommand{\b}{\beta}
\newcommand{\g}{\gamma}
\renewcommand{\d}{\delta}
\newcommand{\e}{\epsilon}
\renewcommand{\k}{\kappa}
\renewcommand{\l}{\lambda}
\renewcommand{\o}{\omega}
\renewcommand{\t}{\theta}
\newcommand{\s}{\sigma}
\newcommand{\p}{\varphi}
\newcommand{\z}{\zeta}
\newcommand{\vt}{\vartheta}
\renewcommand{\O}{\Omega}
\newcommand{\D}{\Delta}
\newcommand{\G}{\Gamma}
\newcommand{\T}{\Theta}
\renewcommand{\L}{\Lambda}

%Mathematical operators
\newcommand{\INT}{\int_{\O}}
\newcommand{\DINT}{\int_{\d\O}}
\newcommand{\Int}{\int_{-\infty}^{\infty}}
\newcommand{\del}{\partial}

\newcommand{\inpr}[2]{\left\langle #1,#2 \right\rangle}
\newcommand{\fr}[2]{\frac{#1}{#2}}
\newcommand{\x}{\times}

\DeclareMathOperator{\dive}{div}
\DeclareMathOperator{\id}{id}
\DeclareMathOperator{\pr}{pr}
\DeclareMathOperator{\Diff}{Diff}
\DeclareMathOperator{\supp}{supp}
\DeclareMathOperator{\graph}{graph}
\DeclareMathOperator{\osc}{osc}
\DeclareMathOperator{\const}{const}
\DeclareMathOperator{\dist}{dist}
\DeclareMathOperator{\loc}{loc}

%Environments
\newcommand{\Theo}[3]{\begin{#1}\label{#2} #3 \end{#1}}
\newcommand{\pf}[1]{\begin{proof} #1 \end{proof}}
\newcommand{\eq}[1]{\begin{equation}\begin{alignedat}{2} #1 \end{alignedat}\end{equation}}
\newcommand{\IntEq}[4]{#1&#2#3	 &\quad &\text{in}~#4,}
\newcommand{\BEq}[4]{#1&#2#3	 &\quad &\text{on}~#4}
\newcommand{\br}[1]{\left(#1\right)}



%Logical symbols
\newcommand{\Ra}{\Rightarrow}
\newcommand{\ra}{\rightarrow}
\newcommand{\hra}{\hookrightarrow}
\newcommand{\mt}{\mapsto}

%Fonts
\newcommand{\mc}{\mathcal}
\renewcommand{\it}{\textit}
\newcommand{\mrm}{\mathrm}

%Spacing
\newcommand{\hp}{\hphantom}


\parindent 0 pt

\protected\def\ignorethis#1\endignorethis{}
\let\endignorethis\relax
\def\TOCstop{\addtocontents{toc}{\ignorethis}}
\def\TOCstart{\addtocontents{toc}{\endignorethis}}

\begin{document}

\title[]
 {Harnack estimates for curvature flow equations in space forms}

\curraddr{}
\email{}
\date{\today}

\dedicatory{}
\subjclass[2010]{}
\keywords{}

\begin{abstract}
%We prove differential Harnack inequalities for weakly convex hypersurfaces on the unit sphere shrinking by positive powers of homogeneous degree one, inverse-concave functions of principal curvatures, $f^{\alpha}>0,~\alpha\geq 1$ . As an application, we prove that the only convex, ancient solutions of curvature flows with $f$ concave and inverse-concave velocities are shrinking geodesic spheres.
\end{abstract}

\maketitle

\section{Introduction}

We consider the evolution of a hypersurface $M^n$ by
\[
\partial_tx=-\varphi\nu,~ x:M^n\times[0,T)\to M_K,
\]
where \(M_K\) is the simply connected space form of constant sectional curvature \(K\) and where $\varphi$ is a function on positive, symmetric linear transformations, evaluated at the Weingarten map,
\[
\varphi = \varphi(\mathcal{W}).
\]
Equivalently, \(\varphi\) is a symmetric function on the eigenvalues of \(\mathcal{W}\) (principal curvatures) \(\kappa_1, \cdots, \kappa_n\).

We are particularly interested in the case \(\varphi = f^{\alpha}\) where \(f\) is homogeneous degree one, inverse concave (concave also?) and \(\alpha \in (0,1)\). Our principal result is a Harnack inequality for these flows, extending the inequality in \cite{2015arXiv150802821B, bryanlouie} and inspired by \cite{MR1316556, MR1100812, MR1296393, MR1480081}. Namely, we prove that if \(\varphi = H^{\alpha}\) for \(0 < \alpha \leq 1\) we have

\begin{theorem}
If $\frac{1}{2}+\frac{1}{2n}\leq {\alpha}< 1,$ then
\[
\partial_t H^{\alpha} - b^{ij}\nabla_iH^{\alpha}\nabla_jH^{\alpha} - \frac{K {\alpha}}{2{\alpha}-1}H^{2{\alpha}-1} + \frac{{\alpha}}{{\alpha}+1} \frac{H^{\alpha}}{t} > 0.
\]
If $0<{\alpha}\leq \frac{1}{2} + \frac{1}{2n}$ or ${\alpha}=1$, then
\[
\partial_t H^{\alpha} - b^{ij}\nabla_iH^{\alpha}\nabla_jH^{\alpha} - K n{\alpha}H^{2{\alpha}-1} + \frac{{\alpha}}{{\alpha}+1} \frac{H^{\alpha}}{t} > 0.
\]
\end{theorem}

\section{Preliminaries}

Let $\overline{g}$ and $\overline{Rm}$ denote, respectively, the metric and the curvature tensor of $M_K$. Define \(M_t := x(M^n,t)\), and let \(g = x_t^{\ast} \overline{g}\) denote the induced metric on \(M\) with $\nabla$ the corresponding Levi-Civita connection. Write $\nu$ for the outer unit normal to $M_t$ and let \(\{\partial_i\}_{i=1}^n\) be a local frame on \(M\) which extends to a frame \(\{\partial_0 = \nu, (x_t)_{\ast} \partial_1, \cdots, (x_t)_{\ast} \partial_n\}\) on \(M_K\) in a neighbourhood of \(M_t\). 

Let Greek indices range from \(0\) to \(n\) and Latin indices range from \(1\) to \(n\). The Riemann curvature tensor of \(M_K\) satisfies \(\bar{R}_{\alpha\beta\gamma\theta} = K(\bar{g}_{\alpha\gamma}\bar{g}_{\beta\delta} - \bar{g}_{\alpha\theta}\bar{g}_{\beta\gamma})\). We may write the metric $g = \{g_{ij}\}$, second fundamental form $A = \{h_{ij}\}$, the Weingarten map $\mathcal{W} = h^i_j = g^{mi} h_{jm}$ and the Riemann curvature tensor $Rm_{ijkl}$ with respect to the given frame. 

We shall write \(\nabla_i\) for covariant derivatives and also use the notation \(\nabla^i = g^{ik} \nabla_k\). Second covariant derivatives will be written \(\nabla^2_{ij} = \nabla_i \nabla_j - \nabla_{\nabla_i \partial j}\) and \((\nabla^2)^i_j = g^{ik} \nabla^2_{kj}\). 

The mean curvature of $M^n$ is the trace of the Weingarten map (equivalently the trace of the second fundamental form with respect to $g$), $H = g^{ij}h_{ij} = h^i_i$. We also use the following standard notation
\[
(h^2)_i^j = g^{mj}g^{rs}h_{ir}h_{sm},
\]
\[
(h^2)_{ij} = g_{kj} (h^2)_i^k = h^k_i h_{kj},
\]
\[
|A|^2 = g^{ij}g^{kl}h_{ik}h_{lj} = h_{ij}h^{ij},
\]
Here, $\{g^{ij}\}$ is the inverse matrix of $\{g_{ij}\}.$ For a strictly convex hypersurface, \(A\) is strictly positive-definite and hence has a strictly positive-definite inverse, which we denote by 
\[
b = \{b^{ij}\}.
\]

The relations between $A$, $Rm$, and $\overline{Rm}$ are given by the Gau{\ss} and Codazzi equations:
\[
\begin{split}
Rm_{ijkl} &= \overline{Rm}_{ijkl} + h_{ik}h_{jl} - h_{il}h_{jk} \\
&= K(\bar{g}_{ik}\bar{g}_{jl} - \bar{g}_{il}\bar{g}_{jk}) + h_{ik}h_{jl} - h_{il}h_{jk},
\end{split}
\]
\[
\nabla_i h_{jk} = \nabla_k h_{ij},
\]
valid for space forms. We also make use of the Ricci identity,
\[
{Rm^m}_{kij}  = \left(\nabla^2_{i, j} \partial_k - \nabla^2_{j,i} \partial_k\right)^m
\]

We will need some notation for derivatives of the speed \(\varphi\). Let us write
\[
\varphi^{i}_{j} = \frac{\partial \varphi}{\partial h^{j}_{i}}
\]
for the first partial derivatives of \(\varphi\). We may also think of \(\varphi\) as a function of the metric and second fundamental form
\[
\varphi(g, h) = \varphi(g^{ik} h_{kj}).
\]
From this point of view, for the first and second partial derivatives, let us write
\[
\varphi^{ij} = \frac{\partial\varphi}{\partial h_{ij}}, \quad \varphi^{ij,kl} = \fr{\partial^2\varphi}{\partial h_{kl} \partial h_{ij}}.
\]
The trace of \(\varphi^{ij}\) with respect to the metric will be written
\[
\Phi = g_{ij} \varphi^{ij}.
\]
Let us also define the operator
\[
\Box = \varphi^{ij} \nabla^2_{ij}
\]
This operator satisfies the product rule
\begin{equation}
\label{eq:productbox}
\Box (fg) = f \Box g + g \Box f + 2 \varphi^{ij} \nabla_i f \nabla_j g.
\end{equation}

We frequently make use, without comment, of the formula for differentiating an inverse
\[
\frac{\partial g^{ij}}{\partial g_{kl}} = - g^{kj} g^{il}.
\]

First derivatives of \(\varphi\) from the two perspectives are related by
\begin{equation}
\label{eq:delh}
\varphi^{ij} = \frac{\partial \varphi}{\partial h_l^k} \frac{\partial h_l^k}{\partial h_{ij}} = \varphi^l_k g^{ik} \delta^j_l = g^{ik} \varphi^j_k.
\end{equation}
and
\begin{equation}
\label{eq:delg}
\frac{\partial\varphi}{\partial g_{ij}} = \varphi^{l}_{k} \frac{\partial h^{k}_{l}}{\partial g_{ij}} = -\varphi^{l}_{k} g^{ki} g^{rj} h_{rl} = -\varphi^{li}h^{j}_{l}.
\end{equation}
We will also need the mixed second derivatives,
\begin{equation}
\label{eq:delhdelg}
\begin{split}
\frac{\partial \varphi^{ij}}{\partial g_{kl}} &= \frac{\partial}{\partial g_{kl}} \left(g^{sj} \varphi^{i}_{s} \right) = - g^{kj}g^{sl} \varphi^{i}_{s} - g^{sj} (\varphi^i_s)^{mk} h^l_m \\
&= - g^{kj} \varphi^{il} - g^{sj} g_{ns} \varphi^{in,mk} h^l_m \\
&= - g^{kj} \varphi^{il} - \varphi^{ij,mk} h^l_m
\end{split}
\end{equation}
where we applied \cref{eq:delg} to \(\varphi^i_s\) in the first line.

Covariant derivatives of \(\varphi\) satisfy
\begin{equation}
\label{eq:delphi}
\nabla_k \varphi = \varphi^{ij} \nabla_k h_{ij}
\end{equation}
and the covariant derivative of the trace,
\begin{equation}
\label{eq:delPhi}
\nabla_k \Phi = g_{ij} \varphi^{ij,rs} \nabla_k h_{rs}.
\end{equation}

\section{Evolution equations}

Following \cite{MR1316556, MR1100812, MR1296393, MR1480081}, in this section, we aim to calculate the evolution of
\[
\chi =t(\partial_t \varphi - K \varphi \Phi - b^{ij} \nabla_i \varphi \nabla_j \varphi) +\delta\varphi
\]
where \(\delta \ne 0\) is an arbitrary, non-zero constant. Let us make a few definitions to keep the calculations more manageable. Let
\[
\alpha_{ij} = \nabla^2_{ij} \varphi + \varphi(h^2)_{ij}, \quad \gamma_{ij} = b^{kl} \nabla_k \varphi \nabla_l h_{ij}, \quad \eta_{ij} = \alpha_{ij} - \gamma_{ij}
\]
and define
\[
\beta = \varphi^{ij} \alpha_{ij} = \Box\varphi + \varphi \varphi^{ij}(h^2)_{ij}, \quad \theta =  b^{ij} \nabla_i \varphi \nabla_j \varphi
\]
so that from the evolution of \(\varphi\) below (\cref{lem:evolution}, \cref{eq:delt_speed}) we may write
\[
\chi = t(\beta - \theta) + \delta\varphi.
\]

We begin by recalling some standard evolution equations and commutators and then break the remaining calculation into several lemmas.

The evolution equations in the following lemma are standard and can be found in many places \cite{MR892052, MR1316556, MR1100812, MR1296393, MR1480081}. The basic tools are commuting derivatives, using the definition of the curvature tensor for space forms, the Gauss equation, and the Codazzi equation as described in the previous section. Compare also \cite[p.~94-95]{Gerhardt:/2006} and the formula \cite[eq.~(6.17)]{Gerhardt:01/1996}.

\begin{lemma}
\label{lem:evolution}
The following evolution equations hold
\begin{enumerate}
\item \label{eq:delt_metric} $\partial_tg_{ij} = -2\varphi h_{ij}$
\item \label{eq:delt_inversemetric} $\partial_t g^{ij} = 2\varphi h^{ij}$
\item \label{eq:delt_christoffel} $\partial_t {\G}^{k}_{ij} = -\varphi g^{kl} \nabla_l h_{ij} - h^k_i \nabla_j \varphi - h^k_j \nabla_i \varphi + g^{kl} h_{ij} \nabla_l \varphi$
\item \label{eq:delt_sff} $\partial_t h_{ij} = \nabla^2_{ji} \varphi - \varphi(h^2)_{ij} + K \varphi g_{ij}$
\item \label{eq:delt_weingarten} $\partial_t h_i^j = (\nabla^2)^j_i\varphi + \varphi(h^2)_i^j + K \varphi\delta_i^j = \alpha^j_i + K \varphi\delta_i^j$
\item \label{eq:delt_sff_box} \begin{align*}
\partial_t h_{ij} &= \Box h_{ij} + \varphi^{kl} (h^2)_{kl} h_{ij} - (\varphi^{kl}h_{kl} + \varphi) (h^2)_{ij} \\
& \quad + \varphi^{kl,rs}\nabla_i h_{kl}\nabla_j h_{rs} \\
& \quad + K \{(\varphi + \varphi^{kl}h_{kl}) g_{ij} - \Phi h_{ij}\}
\end{align*}
\item \label{eq:delt_weingarten_box} \begin{align*}
\partial_t h_i^j &= \Box h_i^j + \varphi^{kl} (h^2)_{kl} h_i^j - (\varphi^{kl}h_{kl} - \varphi) (h^2)_i^j \\
& \quad + \varphi^{kl,rs}\nabla_i h_{kl}\nabla^j h_{rs} \\
& \quad + K \{(\varphi + \varphi^{kl}h_{kl}) \delta_i^j - \Phi h_i^j\}
\end{align*}
\item \label{eq:delt_inversesff} \begin{align*}
\partial_t b^{ij} &= \Box b^{ij} - \varphi^{rs} (h^2)_{rs} b^{ij} + (\varphi^{kl}h_{kl} + \varphi) g^{ij} \\  
& \quad - \left(2b^{lq}\varphi^{kp} + \varphi^{kl,pq}\right) b^{ir}b^{js} \nabla_r h_{kl} \nabla_s h_{pq} \\
& \quad - K \{(\varphi + \varphi^{kl}h_{kl}) b^{ir}b^{j}_{r} - \Phi b^{ij}\}
\end{align*}
\item \label{eq:delt_squaredsff} $\partial_t (h^2)_{ij} = h^k_j \nabla^2_{i,k} \varphi + h^k_i \nabla^2_{j,k} \varphi + h^k_j \varphi(h^2)_{ik} - h^k_i \varphi(h^2)_{jk} + 2K\varphi h_{ij}$
\item \label{eq:delt_speed} $\partial_t \varphi = \Box \varphi + \varphi\varphi^{ij}(h^2)_{ij} + K \varphi\varphi^{ij}g_{ij} = \beta + K\varphi\Phi$
\end{enumerate}
\end{lemma}

We require the commutators \([\nabla, \Box]\) and \([\partial_t, \Box]\). Without further comment we will also use the fact that \([\partial_t, \nabla] f = 0\) for any smooth function \(f\).

\begin{lemma}
\label{lem:gradBox}
For every smooth function $f$, the commutation relation
\[
\begin{split}
([\nabla, \Box]f)_i &= \nabla_i \Box f - (\Box \nabla f)_i = \varphi^{kl,rs} \nabla_i h_{rs} \nabla^2_{kl} f \\
&\quad + K \varphi^{kl} g_{ki} \nabla_l f - K \Phi \nabla_i f \\
&\quad + \varphi^{kl}\left(h^{m}_{l}h_{ki} - h_{kl}h^{m}_{i}\right) \nabla_m f
\end{split}
\]
holds, where \(\nabla f = df\) is the covariant derivative of \(f\) and the subscript \(i\) refers to the \(i\)'th component of a one-form.
\end{lemma}

\begin{proof}
From the Ricci identities
\[
{Rm^m}_{kij}  = \left(\nabla^2_{i, j} \partial_k - \nabla^2_{j,i} \partial_k\right)^m
\]
we obtain that the $3$-tensor $\nabla^3_{kli}f-\nabla^3_{kil}f$
is given by
\[
\nabla^3_{kli}f-\nabla^3_{kil}f={Rm^m}_{kli}\nabla_m f
\]
and thus we obtain
\[
\nabla_i (\varphi^{kl} \nabla^2_{kl} f) - (\varphi^{kl}(\nabla^2_{kl} \nabla f))_i = \varphi^{kl,rs} \nabla_i h_{rs} \nabla^2_{kl}f + \varphi^{kl}{Rm^{m}}_{kli} \nabla_m f.
\]
From the Gauss equation we obtain
\[
\begin{split}
{Rm^{m}}_{kli} \nabla_m f &= \left(K\left(g^{pm}g_{ki}g_{pl}  - g^{pm}g_{pi}g_{kl}\right) + g^{pm} h_{pl}h_{ki} - g^{pm}h_{kl}h_{pi}\right) \nabla_m f \\
&= K\left(g_{ki} \nabla_l f - g_{kl} \nabla_i f\right) + \left(h^{m}_{l}h_{ki} - h_{kl}h^{m}_{i}\right) \nabla_m f.
\end{split}
\]
\end{proof}

\begin{lemma}
\label{lem:deltBox}
The following commutation relation holds
\[
\begin{split}
[\partial_t, \Box] \varphi &= (\partial_{t}\Box - \Box\partial_{t}) \varphi = \varphi^{ij,kl} \nabla^2_{i,j} \varphi (\alpha_{kl} + K \varphi g_{kl}) \\
&\quad + 2\varphi^{ij}h^{k}_{i} (\varphi \nabla^2_{k,j} \varphi + \nabla_k \varphi \nabla_j \varphi) + (\varphi - \varphi^{ij}h_{ij})| \nabla\varphi|^{2}.
\end{split}
\]
\end{lemma}

\begin{proof}
First, let us calculate the evolution of \(\varphi^{ij}\), which will also prove useful later. From the mixed derivative \cref{eq:delhdelg}, the evolution of the metric (\cref{lem:evolution}, \cref{eq:delt_metric}), and the evolution of the second fundamental form (\cref{lem:evolution}, \cref{eq:delt_sff}) we compute
\begin{equation}
\label{eq:deltBox}
\begin{split}
\partial_{t} \varphi^{ij} &= \varphi^{ij,kl} \partial_t h_{kl} + \frac{\partial\varphi^{ij}}{\partial g_{kl}} \partial_t g_{kl} \\
&= \varphi^{ij,kl} \left(\nabla^2_{kl} \varphi - \varphi(h^2)_{kl} + K \varphi g_{kl}\right) + 2\varphi \varphi^{ij,kl} h_{lm}h^{m}_{k} + 2\varphi\varphi^{jk}g^{li}h_{kl} \\
&= \varphi^{ij,kl} \left(\nabla^2_{kl} \varphi + \varphi(h^2)_{kl} + K \varphi g_{kl}\right) + 2\varphi\varphi^{jk}h^{i}_{k} \\
&= \varphi^{ij,kl} \left(\alpha_{kl} + K \varphi g_{kl}\right) + 2\varphi\varphi^{jk}h^{i}_{k}.
\end{split}
\end{equation}

Next, the commutator of \(\partial_t\) and \(\nabla^2_{i,j}\) is given by,
\begin{equation}
\label{eq:deltnabla2}
\begin{split}
\left(\partial_{t}\nabla^2_{i,j} - \nabla^2_{i,j}\partial_{t}\right) \varphi &= \partial_t \left(\nabla_i \nabla_j \varphi - \nabla_{\nabla_i \partial_j} \varphi\right) - \nabla_i \nabla_j \partial_t \varphi + \nabla_{\nabla_i \partial_j} \partial_t \varphi \\
&= - \partial_t \left(\G_{ij}^k \nabla_{\partial_k} \varphi\right) + \G_{ij}^k \nabla_{\partial_k} \partial_t \varphi \\
&= - \nabla_k \varphi \partial_t \G_{ij}^k.
\end{split}
\end{equation}

We obtain from \cref{eq:deltBox}, \cref{eq:deltnabla2}, and the evolution of the Christoffel symbols (\cref{lem:evolution}, \cref{eq:delt_christoffel}),
\[
\begin{split}
\left(\partial_{t}\Box - \Box\partial_{t}\right) \varphi &= \left(\partial_{t}\varphi^{ij}\right) \nabla^2_{i,j} \varphi - \varphi^{ij}\nabla_k \varphi\partial_t \G^{k}_{ij} \\
&= \left[\varphi^{ij,kl} \left(\alpha_{kl} + K \varphi g_{kl}\right) + 2\varphi\varphi^{jk}h^{i}_{k}\right] \nabla^2_{i,j} \varphi \\
&\quad + \varphi^{ij} \nabla_k \varphi \left(\varphi g^{kl} \nabla_l h_{ij} + h^k_i \nabla_j \varphi + h^k_j \nabla_i \varphi - g^{kl} h_{ij} \nabla_l \varphi\right) \\
&= \varphi^{ij,kl} \nabla^2_{i,j} \varphi \left(\alpha_{kl} + K \varphi g_{kl}\right) \\
&\quad + 2\varphi^{jk}h^{i}_{k}\varphi \nabla^2_{i,j} \varphi + h^k_i \varphi^{ij} \nabla_k \varphi \nabla_j \varphi + h^k_j \varphi^{ij} \nabla_k \varphi \nabla_i \varphi \\
&\quad + \varphi g^{kl}\nabla_k \varphi \varphi^{ij} \nabla_l h_{ij} - \varphi^{ij} h_{ij} g^{kl}\nabla_k \varphi \nabla_l \varphi.
\end{split}
\]
The result now follows from \(|\nabla \varphi|^2 = g^{kl}\nabla_k \varphi \nabla_l \varphi\) and \(\nabla_l \varphi = \varphi^{ij} \nabla_l h_{ij}\) by the chain rule.
\end{proof}

The next ingredient is the evolution of the covariant derivative, \(\nabla \varphi = d\varphi\).

\begin{lemma}
\label{lem:Evgradphi}
There holds
\[
\begin{split}
\left((\partial_{t}-\Box)\nabla\varphi\right)_{i} &= \varphi^{kl,rs}\nabla_i h_{rs} \alpha_{kl} + 2 \varphi^{kl} b^{rs} \varphi(h^2)_{rl} \nabla_i h_{ks} \\
&\quad + \varphi^{kl}(h^2)_{kl}\nabla_i \varphi + \left(\varphi^{kl}h^{m}_{l}h_{ki} - \varphi^{kl}h_{kl}h^{m}_{i}\right) \nabla_m \varphi\\
&\quad + K\left(\varphi^{kl}g_{ki} \nabla_l \varphi + \varphi \nabla_i \Phi\right).
\end{split}
\]
\end{lemma}

\begin{proof}
Using the evolution of \(\varphi\) \cref{lem:evolution}, \cref{eq:delt_speed}, the derivative of \(\Phi\) \cref{eq:delPhi}, and the commutator \([\nabla, \Box]\) from \cref{lem:gradBox}, we compute
\[
\begin{split}
\partial_{t}\nabla_i \varphi - (\Box\nabla \varphi)_{i} &= \nabla_i \partial_t \varphi - \nabla_i \Box \varphi + ([\nabla, \Box] \varphi)_i \\
&= \nabla_i \left(\Box\varphi + \varphi^{kl}(h^2)_{kl}\varphi + K \Phi\varphi\right) - \nabla_i (\Box\varphi) \\
&\quad + \varphi^{kl,rs} \nabla_i h_{rs} \nabla^2_{kl} \varphi + K\varphi^{kl}g_{ki} \nabla_l \varphi - K\Phi\nabla_i \varphi \\
&\quad + (\varphi^{kl}h^{m}_{l}h_{ki} - \varphi^{kl}h_{kl}h^{m}_{i}) \nabla_m \varphi \\
&= \varphi^{kl}(h^2)_{kl}\nabla_i \varphi + \varphi^{kl,rs}\nabla_i h_{rs} (h^2)_{kl}\varphi + \varphi\varphi^{kl}(h^s_l \nabla_i h_{ks} + h^r_k \nabla_i h_{rl}) + K \varphi\nabla_i\Phi \\
&\quad + \varphi^{kl,rs} \nabla_i h_{rs} \nabla^2_{kl} \varphi + K\varphi^{kl}g_{ki}\nabla_l \varphi \\
&\quad + (\varphi^{kl}h^{m}_{l}h_{ki} - \varphi^{kl}h_{kl}h^{m}_{i}) \nabla_m \varphi \\
&= \varphi^{kl,rs}\nabla_i h_{rs} \left((h^2)_{kl}\varphi + \nabla^2_{kl} \varphi\right) + 2 \varphi\varphi^{kl} h^s_l \nabla_i h_{ks} \\
&\quad + \varphi^{kl}(h^2)_{kl}\nabla_i \varphi + (\varphi^{kl}h^{m}_{l}h_{ki} - \varphi^{kl}h_{kl}h^{m}_{i}) \nabla_m \varphi \\
&\quad + K \left(\varphi\nabla_i\Phi + \varphi^{kl}g_{ki}\nabla_l \varphi\right).
\end{split}
\]
where in the third equality, we used
\[
\nabla_i (h^2)_{kl} = \nabla_i (g^{sr} h_{ks} h_{rl}) = h^s_l \nabla_i h_{ks} + h^r_k \nabla_i h_{rl}
\]
and in the last equality we used
\[
b^{rs} (h^2)_{rl} = b^{rs} h_{rm} h^m_l = \delta^s_m h^m_l = h^s_l.
\]
\end{proof}

Now we may proceed to the calculation of \(\partial_t \chi\). But we will do this first by making the intermediate calculations of \(\partial_t \beta\) and \(\partial_t \theta\).

\begin{lemma}
\label{lem:evbeta}
The quantity 
\[
\beta = \partial_t \varphi - K\Phi\varphi = \Box\varphi +  \varphi\varphi^{ij} (h^2)_{ij}
\]
satisfies
\[
\begin{split}
(\partial_{t} - \Box)\beta &= \left(\varphi^{ij}(h^2)_{ij} + K\Phi \right)\beta \\
&\quad + (\varphi - \varphi^{ij}h_{ij}) |\nabla\varphi|^{2} + 2\varphi^{ij}h^{k}_{i}\nabla_k \varphi \nabla_j \varphi \\
&\quad + \varphi^{ij,kl} \alpha_{ij} \alpha_{kl} \\
&\quad + 2b^{il}\varphi^{jk} (2\nabla^2_{ij}\varphi\varphi(h^2)_{kl} + \varphi(h^2)_{ij}\varphi(h^2)_{kl}) \\
&\quad + KR_{\beta}
\end{split}
\]
where
\[
R_{\beta} = \varphi \Box \Phi + 2\varphi^{kl} \nabla_k \Phi \nabla_l \varphi + \varphi \varphi^{ij,kl}g_{kl} \alpha_{ij} + 2\varphi^{2}\varphi^{ij}h_{ij}.
\]
\end{lemma}

\begin{proof}
Let us break up the calculation of
\[
(\partial_{t} - \Box)\beta =  \partial_{t}\Box\varphi + \partial_{t} (\varphi\varphi^{ij} (h^2)_{ij}) - \Box\Box\varphi - \Box(\varphi\varphi^{ij} (h^2)_{ij})
\]
into smaller pieces. First, we have lots of nice cancellation. Using the evolution of \(\varphi\) from \cref{lem:evolution}, \cref{eq:delt_speed} and the commutator relation from \cref{lem:deltBox} we have,
\begin{equation}
\label{eq:deltbeta1}
\begin{split}
\partial_{t}\Box\varphi - \Box\Box\varphi - \Box(\varphi\varphi^{ij} (h^2)_{ij}) &= \Box\partial_t\varphi - \Box\Box\varphi - \Box(\varphi\varphi^{ij} (h^2)_{ij}) + [\partial_t, \Box] \varphi \\
&= \Box(\Box \varphi + \varphi\varphi^{ij}(h^2)_{ij} + K \varphi\Phi) - \Box\Box\varphi - \Box(\varphi\varphi^{ij} (h^2)_{ij}) \\
&\quad + \varphi^{ij,kl} \nabla^2_{i,j} \varphi (\alpha_{kl} + K \varphi g_{kl}) \\
&\quad + 2\varphi^{ij}h^{k}_{i} (\varphi \nabla^2_{k,j} \varphi + \nabla_k \varphi \nabla_j \varphi) + (\varphi - \varphi^{ij}h_{ij})| \nabla\varphi|^{2} \\
&= K (\Phi \Box \varphi + \varphi \Box \Phi + 2 \varphi^{kl} \nabla_k \Phi \nabla_l \varphi) \\
&\quad + \varphi^{ij,kl} \nabla^2_{ij} \varphi (\alpha_{kl} + K \varphi g_{kl}) \\
&\quad + 2\varphi^{ij}b^{kl} \varphi (h^2)_{il} \nabla^2_{kj} \varphi + 2\varphi^{ij}h^{k}_{i} \nabla_k \varphi \nabla_j \varphi + (\varphi - \varphi^{ij}h_{ij})| \nabla\varphi|^{2} \\
&= K \Phi \Box \varphi  \\
&\quad + (\varphi - \varphi^{ij}h_{ij})| \nabla\varphi|^{2} \\
&\quad + \varphi^{ij,kl} \nabla^2_{ij} \varphi \alpha_{kl} \\
&\quad + 2\varphi^{ij}b^{kl} \varphi (h^2)_{il} \nabla^2_{kj} \varphi + 2\varphi^{ij}h^{k}_{i} \nabla_k \varphi \nabla_j \varphi \\
&\quad + K(\varphi \Box \Phi + 2 \varphi^{kl} \nabla_k \Phi \nabla_l \varphi + \varphi \varphi^{ij,kl} g_{kl} \nabla^2_{ij} \varphi)
\end{split}
\end{equation}
using, in the third equality, the product rule for \(\Box\) \cref{eq:productbox} and \(h^k_i = h^m_i b^{kl}h_{ml} = b^{kl} (h^2)_{il}\) since \(b\) is the inverse of \(A\).

Next from \cref{eq:deltBox} and \cref{lem:evolution}, \cref{eq:delt_squaredsff} we obtain
\[
\begin{split}
\partial_{t} (\varphi^{ij}(h^2)_{ij}) &= \partial_{t}(\varphi^{ij}) (h^2)_{ij} + \varphi^{ij} \partial_t (h^2)_{ij} \\
&= \left(\varphi^{ij,kl} \left(\alpha_{kl} + K \varphi g_{kl}\right) + 2\varphi\varphi^{jk}h^{i}_{k}\right) (h^2)_{ij} \\
&\quad + \varphi^{ij} \left(h^k_j \nabla^2_{i,k} \varphi + h^k_i \nabla^2_{j,k} \varphi + h^k_j \varphi(h^2)_{ik} - h^k_i \varphi(h^2)_{jk} + 2K\varphi h_{ij}\right) \\
&= \varphi^{ij,kl} (h^2)_{ij} \left(\alpha_{kl} + K \varphi g_{kl}\right) + 2\varphi\varphi^{jk} b^{il} (h^2)_{lk} (h^2)_{ij} \\
&\quad + 2 \varphi^{ij} b^{kl} (h^2)_{il} \nabla^2_{j,k} \varphi  + 2K\varphi\varphi^{ij}h_{ij}
\end{split}
\]
again using \(h^k_i = b^{kl} (h^2)_{il}\) in the last equality.

The remaining term we need to compute is thus
\begin{equation}
\label{eq:deltbeta2}
\begin{split}
\partial_{t} (\varphi \varphi^{ij}(h^2)_{ij}) &= (\partial_{t} \varphi) \varphi^{ij}(h^2)_{ij} + \varphi \partial_t (\varphi^{ij} (h^2)^{ij}) \\
&= (\beta + K \varphi\Phi) \varphi^{ij}(h^2)_{ij} \\
&\quad + \varphi \left[\varphi^{ij,kl} (h^2)_{ij} \left(\alpha_{kl} + K \varphi g_{kl}\right) + 2\varphi\varphi^{jk} b^{il} (h^2)_{lk} (h^2)_{ij} \right.\\
&\quad \left. + 2 \varphi^{ij} b^{kl} (h^2)_{il} \nabla^2_{jk} \varphi  + 2K\varphi\varphi^{ij}h_{ij}\right]. \\
&= (\beta + K \varphi\Phi) \varphi^{ij}(h^2)_{ij} \\
&\quad + \varphi^{ij,kl} \varphi (h^2)_{ij} \alpha_{kl}  \\
&\quad + 2 \varphi\varphi^{ij} b^{kl} (h^2)_{il} \nabla^2_{jk} \varphi + 2\varphi^{jk} b^{il} \varphi (h^2)_{lk} \varphi (h^2)_{ij} \\
&\quad + K \left[\varphi \varphi^{ij,kl} g_{kl} \varphi (h^2)_{ij} + 2K\varphi^2\varphi^{ij}h_{ij}\right]
\end{split}
\end{equation}

Now we add \cref{eq:deltbeta1} and \cref{eq:deltbeta2} together line by line to complete the proof.
\end{proof}

\begin{lemma}
\label{lem:Evtheta}
The quantity $\theta = b^{ij}\nabla_i \varphi\nabla_j\varphi$ satisfies
\[
\begin{split}
(\partial_{t} - \Box)\theta &= (\varphi^{ij}(h^2)_{ij} + K\Phi)\theta \\
&\quad + (\varphi - \varphi^{ij}h_{ij})|\nabla\varphi|^{2} + 2\varphi^{ij}h^{k}_{i}\nabla_k\varphi\nabla_j\varphi \\
&\quad - \varphi^{kl,ij} (\gamma_{ij}\gamma_{kl}  - 2\alpha_{ij} \gamma_{kl}) \\
&\quad - 2b^{il} \varphi^{jk} \left(\gamma_{ij} \gamma_{kl} - 2\alpha_{ij} \gamma_{kl} + \nabla^2_{ij}\varphi\nabla^2_{kl}\varphi\right) \\
&\quad + KR_{\theta}.
\end{split}
\]
\end{lemma}
where
\[
R_{\theta} = -(\varphi^{kl}h_{kl} + \varphi)b^{ir}b^{j}_{r}\nabla_i \varphi\nabla_j\varphi + 2 b^{j}_{k}\varphi^{kl}\nabla_l\varphi\nabla_j\varphi + 2 \varphi\varphi^{ij,kl} g_{ij} \gamma_{kl}.
\]

\begin{proof}
Again using the product rule for \(\Box\), \cref{eq:productbox} and the symmetry \(b^{ij} = b^{ji}\), we have
\begin{equation}
\label{eq:delt_theta}
\begin{split}
(\partial_{t} - \Box)\theta &= (\partial_{t}b^{ij} - \Box b^{ij})\nabla_i \varphi\nabla_j\varphi + b^{ij} (\partial_{t} - \Box) (\nabla\varphi \otimes \nabla\varphi)_{ij} \\
&\quad - 2 \varphi^{kl} \nabla_k b^{ij} \nabla_l (\nabla \varphi \otimes \nabla\varphi)_{ij} \\
&= (\partial_{t}b^{ij} - \Box b^{ij})\nabla_i \varphi\nabla_j\varphi + 2 b^{ij} (\partial_{t} - \Box) (\nabla\varphi)_i \nabla_j\varphi - 2 b^{ij} \varphi^{kl} \nabla^2_{i,k} \varphi \nabla^2_{l,j} \varphi \\
&\quad - 4 \varphi^{kl} \nabla_k b^{ij} \nabla^2_{i,l} \varphi \nabla_j\varphi \\
&= (\partial_{t}b^{ij} - \Box b^{ij})\nabla_i \varphi\nabla_j\varphi + 2 b^{ij} (\partial_{t} - \Box) (\nabla\varphi)_i \nabla_j\varphi \\
&\quad - 2 b^{ij} \varphi^{kl} \nabla^2_{i,k} \varphi \nabla^2_{l,j} \varphi + 4 \varphi^{kl} b^{ip}b^{jq} \nabla_k h_{pq} \nabla^2_{i,l} \varphi \nabla_j\varphi \\
&= (\partial_{t}b^{ij} - \Box b^{ij})\nabla_i \varphi\nabla_j\varphi + 2 b^{ij} (\partial_{t} - \Box) (\nabla\varphi)_i \nabla_j\varphi \\ 
&\quad - 2 b^{ij} \varphi^{kl} \nabla^2_{i,k} \varphi \nabla^2_{l,j} \varphi + 4 \varphi^{kl} b^{ip}\gamma_{pk} \nabla^2_{i,l} \varphi
\end{split}
\end{equation}
where in the second to last equality we used the formula for the derivative of the inverse \(b^{ij}\) of \(h_{ij}\) and the Codazzi equation in the last line, producing \(b^{jq} \nabla_k h_{pq} \nabla_j \varphi = b^{jq} \nabla_q h_{pk} \nabla_j \varphi = \gamma_{pk}\). The first term in final line appears on the second to last line of the statement of the lemma (with indices relabelled). The second term is part of \(4 b^{il}\varphi^{jk} \alpha_{ij} \gamma_{kl}\) in the second to last line. So we must deal with the first two terms and show they add to the remainder of the statement. For the first term, we use the evolution of \(b^{ij}\) from \cref{lem:evolution}, \cref{eq:delt_inversesff} to calculate
\begin{equation}
\label{eq:delt_theta1}
\begin{split}
(\partial_{t}b^{ij} - \Box b^{ij})\nabla_i \varphi\nabla_j\varphi &= \nabla_i \varphi \nabla_j \varphi\left[-\varphi^{rs} (h^2)_{rs} b^{ij} + (\varphi^{kl}h_{kl} + \varphi) g^{ij} \right. \\
& \quad - \left(2b^{lq}\varphi^{kp} + \varphi^{kl,pq}\right) b^{ir}b^{js} \nabla_r h_{kl} \nabla_s h_{pq} \\
& \quad - \left. K \{(\varphi + \varphi^{kl}h_{kl}) b^{ir}b^{j}_{r} - \Phi b^{ij}\}\right] \\
&= \left(K\Phi - \varphi^{rs}(h)^2_{rs}\right) \theta - (2b^{lq}\varphi^{kp} + \varphi^{kl,pq}) b^{ir}b^{js}\nabla_i\varphi\nabla_j\varphi\nabla_rh_{kl}\nabla_s h_{pq} \\
&\quad + (\varphi^{kl}h_{kl} + \varphi)|\nabla\varphi|^{2} \\
&\quad - K(\varphi^{kl}h_{kl} + \varphi)b^{ir}b^{j}_{r}\nabla_i \varphi\nabla_j\varphi \\
&= \left(K\Phi - \varphi^{rs}(h)^2_{rs}\right) \theta \\
&\quad + (\varphi^{kl}h_{kl} + \varphi)|\nabla\varphi|^{2} \\
&\quad - \varphi^{kl,pq} \gamma_{kl} \gamma_{pq} \\
&\quad - 2b^{lq}\varphi^{kp} \gamma_{kl} \gamma_{pq} \\
&\quad - K(\varphi^{kl}h_{kl} + \varphi)b^{ir}b^{j}_{r}\nabla_i \varphi\nabla_j\varphi
\end{split}
\end{equation}
For the second term, from the evolution of \(\nabla\varphi\) in \cref{lem:Evgradphi}, we have
\begin{equation}
\label{eq:delt_theta2}
\begin{split}
2 b^{ij} (\partial_{t} - \Box) (\nabla\varphi)_i \nabla_j\varphi &= 2 b^{ij} \nabla_j\varphi \left[\varphi^{kl,rs}\nabla_i h_{rs} \alpha_{kl}\varphi + 2 \varphi^{kl} b^{rs} \varphi(h^2)_{rl} \nabla_i h_{ks} \right. \\
&\quad + \varphi^{kl}(h^2)_{kl}\nabla_i \varphi + \left(\varphi^{kl}h^{m}_{l}h_{ki} - \varphi^{kl}h_{kl}h^{m}_{i}\right) \nabla_m \varphi\\
&\quad \left. + K\left(\varphi^{kl}g_{ki} \nabla_l \varphi + \varphi \nabla_i \Phi\right)\right] \\
&= 2 b^{ij} \nabla_j\varphi \varphi^{kl}(h^2)_{kl}\nabla_i \varphi \\
&\quad - 2 b^{ij} \nabla_j\varphi \varphi^{kl}h_{kl}h^{m}_{i} \nabla_m \varphi + 2 b^{ij} \nabla_j\varphi \varphi^{kl}h^{m}_{l}h_{ki} \nabla_m \varphi \\
&\quad + 2 b^{ij} \nabla_j\varphi \varphi^{kl,rs}\nabla_i h_{rs} \alpha_{kl} \\
&\quad + 4 b^{ij} \nabla_j\varphi \varphi^{kl} b^{rs} \varphi(h^2)_{rl} \nabla_i h_{ks} \\
&\quad + K\left(2 b^{ij} \nabla_j\varphi \varphi^{kl}g_{ki} \nabla_l \varphi + 2 b^{ij} \nabla_j\varphi \varphi \nabla_i \Phi\right) \\
&= 2 \varphi^{kl}(h^2)_{kl}\theta \\
&\quad - 2 \varphi^{kl}h_{kl} |\nabla\varphi|^2 + 2 \varphi^{kl} h^{m}_{l} \nabla_k\varphi \nabla_m \varphi \\
&\quad + 2 \varphi^{kl,rs} \gamma_{rs} \alpha_{kl} \\
&\quad + 4 b^{rs} \varphi^{kl} \gamma_{ks} \varphi(h^2)_{rl} \\
&\quad + K\left(2 \varphi^{kl} b^j_k \nabla_j\varphi \nabla_l \varphi + 2 \varphi b^{ij} \nabla_j\varphi \nabla_i \Phi\right)
\end{split}
\end{equation}
using the definitions of \(\theta, \alpha_{ij}\) and \(\gamma_{ij}\) as well as \(b^{ij}h_{ki} \nabla_j \varphi = \delta^j_k \nabla_j \varphi = \nabla_k \varphi\), and \(b^{ij} h^m_i = b^{ij} g^{mp}h_{pi} = \delta^j_p g^{mp} = g^{mj}\) in the last equality.

The proof is now completed by adding \cref{eq:delt_theta1} and \cref{eq:delt_theta2} line by line and adding also the final line from \cref{eq:delt_theta}.
\end{proof}

\begin{theorem}
\label{thm:Evchi}
Let $\delta \neq 0.$ The quantity 
\[
\chi = t(\beta - \theta) + \delta\varphi
\]
satisfies
\[
\begin{split}
\partial_t \chi -\Box\chi &= \left(\frac{\beta - \theta}{\delta\varphi} + \varphi^{ij}(h)^2_{ij} + K\Phi\right)\chi \\
& \quad + t\left(\varphi^{ij,kl} + 2b^{il}\varphi^{jk} - \frac{\varphi^{ij}\varphi^{kl}}{\delta\varphi}\right)\eta_{ij}\eta_{kl} + tK R,
\end{split}
\]
where
\[
\eta_{ij} = \alpha_{ij} - \gamma_{ij} = \nabla^2_{i,j}\varphi + (h)^2_{ij}\varphi - b^{rs}\nabla_r h_{ij}\nabla_s \varphi
\]
and
\[
\begin{split}
R &= R_{\beta} - R_{\theta} \\
&= \varphi \Box \Phi + 2\varphi^{kl} \nabla_k \Phi \nabla_l \varphi + \varphi \varphi^{ij,kl}g_{kl} \alpha_{ij} + 2\varphi^{2}\varphi^{ij}h_{ij} \\
&\quad +(\varphi^{kl}h_{kl} + \varphi)b^{ir}b^{j}_{r}\nabla_i \varphi\nabla_j\varphi - 2 b^{j}_{k}\varphi^{kl}\nabla_l\varphi\nabla_j\varphi - 2 \varphi\varphi^{ij,kl} g_{ij} \gamma_{kl}
\end{split}
\]

\end{theorem}
	
\begin{proof}
We have
\[
(\partial_t - \Box)\chi = \beta - \theta + t(\partial_{t} - \Box)(\beta - \theta) + \delta(\partial_t \varphi - \Box\varphi).
\]
First of all, the evolution equation for \(\varphi\), \cref{lem:evolution}, \cref{eq:delt_speed} gives us
\[
\delta(\partial_t \varphi - \Box\varphi) = \left(\varphi^{ij}(h)^2_{ij} + K\varphi^{ij}g_{ij}\right)\delta\varphi.
\]
Next, we note that
\[
\varphi^{ij} \eta_{ij} = \beta - \theta
\]
since \(\nabla_r \varphi = \varphi^{ij} \nabla_r h_{ij}\). Putting the two equations above together gives
\begin{equation}
\label{eq:deltchi1}
\begin{split}
\beta - \theta + \delta(\partial_t \varphi - \Box\varphi) &= \left(\frac{\beta-\theta}{\delta\varphi} + \varphi^{ij}(h)^2_{ij} + K\varphi^{ij}g_{ij}\right)\delta\varphi + t \frac{(\beta - \theta)^2}{\delta\varphi} - t \frac{\varphi^{ij}\eta_{ij} \varphi^{kl}\eta_{kl}}{\delta\varphi} \\
&= \frac{\beta-\theta}{\delta\varphi} \chi + \left(\varphi^{ij}(h)^2_{ij} + K\varphi^{ij}g_{ij}\right)\delta\varphi \\
&\quad - t \frac{\varphi^{ij}\varphi^{kl}}{\delta\varphi} \eta_{ij}\eta_{kl}.
\end{split}
\end{equation}

The remaining term \(t(\partial_{t} - \Box)(\beta - \theta)\) is now just bookkeeping. Recall, \cref{lem:evbeta} states that
\begin{align*}
(\partial_{t} - \Box)\beta &= \left(\varphi^{ij}(h^2)_{ij} + K\Phi \right)\beta  & (A) \\
&\quad + (\varphi - \varphi^{ij}h_{ij}) |\nabla\varphi|^{2} + 2\varphi^{ij}h^{k}_{i}\nabla_k \varphi \nabla_j \varphi  & (B) \\
&\quad + \varphi^{ij,kl} \alpha_{ij} \alpha_{kl} & (C) \\
&\quad + 2b^{il}\varphi^{jk} (2\nabla^2_{ij}\varphi\varphi(h^2)_{kl} + \varphi(h^2)_{ij}\varphi(h^2)_{kl}) & (D) \\
&\quad + KR_{\beta}  & (E) \\
\intertext{while \cref{lem:Evtheta} states that}
(\partial_{t} - \Box)\theta &= (\varphi^{ij}(h^2)_{ij} + K\Phi)\theta & (A') \\
&\quad + (\varphi - \varphi^{ij}h_{ij})|\nabla\varphi|^{2} + 2\varphi^{ij}h^{k}_{i}\nabla_k\varphi\nabla_j\varphi & (B') \\
&\quad - \varphi^{kl,ij} (\gamma_{ij}\gamma_{kl}  - 2\alpha_{ij} \gamma_{kl}) & (C') \\
&\quad - 2b^{il} \varphi^{jk} \left(\gamma_{ij} \gamma_{kl} - 2\alpha_{ij} \gamma_{kl} + \nabla^2_{ij}\varphi\nabla^2_{kl}\varphi\right) & (D') \\
&\quad + KR_{\theta} & (E').
\end{align*}

Subtracting line by line, we have
\begin{align*}
(A) - (A') &= \left(\varphi^{ij}(h^2)_{ij} + K\Phi \right)\beta - \left(\varphi^{ij}(h^2)_{ij} + K\Phi \right)\theta = \left(\varphi^{ij}(h^2)_{ij} + K\Phi \right)(\beta - \theta) \\
(B) - (B') &= (\varphi - \varphi^{ij}h_{ij}) |\nabla\varphi|^{2} + 2\varphi^{ij}h^{k}_{i}\nabla_k \varphi \nabla_j \varphi - (\varphi - \varphi^{ij}h_{ij})|\nabla\varphi|^{2} - 2\varphi^{ij}h^{k}_{i}\nabla_k\varphi\nabla_j\varphi = 0 \\
(C) - (C') &= \varphi^{ij,kl} \alpha_{ij} \alpha_{kl} + \varphi^{kl,ij} (\gamma_{ij}\gamma_{kl}  - 2\alpha_{ij} \gamma_{kl}) = \varphi^{ij,kl} (\alpha_{ij} - \gamma_{ij}) (\alpha_{kl} - \gamma_{kl}) = \varphi^{ij,kl} \eta_{ij} \eta_{kl} \\
(D) - (D') &= 2b^{il}\varphi^{jk} (2\nabla^2_{ij}\varphi\varphi(h^2)_{kl} + \varphi(h^2)_{ij}\varphi(h^2)_{kl}) \\
&\quad + 2b^{il} \varphi^{jk} \left(\gamma_{ij} \gamma_{kl} - 2\alpha_{ij} \gamma_{kl} + \nabla^2_{ij}\varphi\nabla^2_{kl}\varphi\right) \\
&= 2b^{il}\varphi^{jk} \left(\nabla^2_{ij}\varphi\nabla^2_{kl}\varphi +2\nabla^2_{ij}\varphi\varphi(h^2)_{kl} + \varphi(h^2)_{ij}\varphi(h^2)_{kl} - 2 \alpha_{ij} \gamma_{kl} + \gamma_{ij} \gamma_{kl} \right) \\
&= 2b^{il}\varphi^{jk} \left(\alpha_{ij}\alpha_{kl} - 2 \alpha_{ij} \gamma_{kl} + \gamma_{ij} \gamma_{kl} \right) = 2b^{il}\varphi^{jk} \eta_{ij} \eta_{kl} \\
(E) - (E') &= K(R_{\beta} - R_{\theta}) = KR\\
\end{align*}

Multiplying everything by \(t\) and adding the result to \cref{eq:deltchi1} gives the result.
\end{proof}
	
\section{Curvature estimates and backwards convergence to an equator}
For a convex $H^{\a}$-flow satisfying a Harnack inequality we obtain $C^2$-estimates easily. For a curvature function $\p$ let us call a hypersurface $M$ $\p$-convex, if at every point $x\in M$ we have
\[\p(\k_1(x),\dots,\k_n(x))\geq 0.
\]

\begin{lemma}\label{SpeedEst}
On some open interval $(a,b)$ let $M_t,$ $t\in(a,b),$ be a $\p$-convex solution of the curvature flow equation
\begin{equation}\label{eq:CurvFlow}
\dot{x}=-\varphi\nu
\end{equation}
for some strictly monotone curvature function $\varphi$ in a spaceform of nonnegative sectional curvature $K\geq 0.$
Then 
\[\tilde{\varphi}=\inf_{M_t}\varphi
\]
is nondecreasing in time.
\end{lemma}

\begin{proof}
$\tilde{\varphi}$ is Lipschitz continuous, differentiable almost everywhere and 
\begin{equation}
\dot{\tilde{\varphi}}(t)=\fr{\del\varphi}{\del t}(t,x_t),
\end{equation}
where $x_t\in M_t$ is a point where the infimum is attained, compare \cite[Lemma 6.3.2]{Gerhardt:/2006}.
By \eqref{eq:delt_speed} we obtain
\eq{\dot{\tilde{\varphi}}\geq 0.}
\end{proof}


%%%%NOTE: For this corollary we should first transform our differential Harnack inequality into an actual estimate. I formulate the following as general as possible, also to cover other cases, e.g. in the Euclidean space. We might want to change this later. Whether the assumptions hold will depend on the specific curvature function. They certainly hold for our H^k flows.
\begin{corollary}
In the situation of Lemma \ref{SpeedEst} with a $k$-homogeneous curvature function $\p$ suppose the $M_t$ are convex and that $\p$ satisfies a Harnack inequality of the form
\eq{\sup_{M_t}\varphi\leq c\inf_{M_{t}}\varphi,}
with a time independent constant $c.$
Furthermore suppose there holds an estimate of the form
\eq{\k_n\leq c\p^{\fr 1k}.}
Then the principal curvatures of the flow hypersurfaces $M_t,$
\eq{\k_1\leq \dots\leq \k_n,}
are uniformly bounded backwards in time, i.e. for all $T\in (a,b)$ there exists a constant $c=c(T)>0,$ such that 
\eq{\k_n\leq c\quad\forall t\in (a,T).}
In particular $c$ does not depend on $a.$
\end{corollary}

\begin{proof}
By Lemma \ref{SpeedEst} the infimum of $\varphi$ is bounded backwards in time and by the Harnack inequality also its supremum is bounded backwards in time. Due to convexity and 
\eq{\k_n\leq c\p^{\fr 1k}}
we obtain the result.
\end{proof}

Let us investigate the behavior of geodesic spheres under the $f^{\a}$-flow in the sphere.

\begin{lemma}
Let $\p=f^{\a},$ $0<\a<1,$ where $f$ is a strictly monotone and $1$-homogeneous curvature function normalized to $f(1,\dots,1)=n.$ Consider \eqref{eq:CurvFlow} in the sphere, i.e. $K=1.$ Then a flow of strictly convex geodesic spheres has a finite lifespan, i.e. let $S_r(p)$ be a geodesic sphere in $\S^{n+1}$ around $p\in \S^{n+1}$ with outward normal $\nu,$ then a strictly convex flow \eqref{eq:CurvFlow} with initial hypersurface $M_0=S_r$ exists at most on a finite time interval $(a,b).$ Here $b$ is the collapsing time and $a$ is the time where the backward flow hits the equator. 
\end{lemma}

\begin{proof}
Starting point is again \eqref{eq:delt_speed}. Since $f$ is constant on a geodesic sphere, for a spherical flow the evolution equation for $\p$ yields
\eq{\fr{d}{dt}f^{\a}\geq \a n f^{2\a-1},}
where we used
\eq{\p^{ij}=\p'f^{ij}}
and 
\eq{f^{ij}=g^{ij}}
at umbilical points, where the latter holds due to the symmetry of $f.$
This yields
\eq{\fr{d}{dt}f\geq nf^{\a}.}
Since the right hand side remains strictly positive under this ODE we obtain finite lifespan forward in time.
Convexity and integration over some interval $(a,b)$ yield 
\eq{0\leq f^{1-\a}(a)\leq f^{1-\a}(b)-(1-\a)n(b-a).}
Letting $a\ra-\8$ gives finite existence backwards in time. 
\end{proof}

\begin{remark}
Since a smooth convex hypersurface $M\sub\S^{n+1}$ is either an equator or strictly contained in an open hemisphere due to the classical paper \cite{CarmoWarner:/1970}, by the avoidance principle any non-equatorial convex solution to \eqref{eq:CurvFlow} collapses in finite time. By a time shift we may from now on suppose that all convex solutions collapse at time $0.$ We discard the trivial equatorial solutions from our investigations.   
\end{remark}

\bibliographystyle{amsplain}
\bibliography{Bibliography.bib}


\end{document}
